
% -------------------------
% Configuración específica de LAB 02
% -------------------------
%

% Redefinir variables del laboratorio
\renewcommand{\labnumber}{LABORATORIO N° 02 - AJUSTE LINEAL}
\renewcommand{\labtitle}{PRINCIPIO DE MÍNIMOS CUADRADOS - AJUSTE LINEAL}
\renewcommand{\labauthor}{Apellidos y Nombres}
\renewcommand{\labcourse}{\textbf{\textit{Curso:}} Física I - Unidad 1}
\renewcommand{\labcodigo}{\textbf{\textit{Código:}} 182740}
\renewcommand{\labinstitution}{Universidad Nacional Micaela Bastidas de Apurimac}
\renewcommand{\labdate}{03 de Noviembre del 2025}
\renewcommand{\labdocente}{\textbf{DOCENTE:} HUALLPA AIMITUMA Josué David}

% Reiniciar numeración de secciones y subsecciones
\setcounter{section}{0}
\setcounter{subsection}{0}
\renewcommand{\thesection}{\arabic{section}}
\renewcommand{\thesubsection}{\thesection.\arabic{subsection}}

% -------------------------
% Cabecera y pie de página
% -------------------------
\pagestyle{fancy}
\fancyhead[R]{\footnotesize \textit{\labtitle}}
\fancyfoot[C]{}
\fancyfoot[LE]{\footnotesize \textbf{\thepage}\hspace{10pt}\labcourse}
\fancyfoot[RO]{\footnotesize \labcourse\hspace{10pt}\textbf{\thepage}}
\fancyfoot[RE,LO]{\footnotesize\itshape \labauthor}

\renewcommand{\headrulewidth}{0pt}  
\renewcommand{\footrulewidth}{0pt}  

\fancypagestyle{firststyle}{
    \fancyhead{} 
    \fancyfoot[R]{\footnotesize \labinstitution\hspace{10pt}\textbf{\labdate}\hspace{10pt}\textbf{\thepage\textendash\pageref{LastPage}}}
    \fancyfoot[L]{\footnotesize\labcourse}
}

% -------------------------
% Portada del laboratorio
% -------------------------
\begin{flushleft}
    {\sffamily\bfseries\fontsize{8}{14}\labnumber\par\vskip8pt}
    {\bfseries\color{labblue}\sffamily\fontsize{18}{22}\selectfont\labtitle}\par\vskip8pt
    {\bfseries\fontsize{11pt}{12pt}\selectfont\sffamily \labauthor}\par\vskip8pt
    {\fontsize{7pt}{8pt}\selectfont\sffamily \labcodigo}\par
    {\fontsize{7pt}{8pt}\selectfont\sffamily \labcourse}\par \vskip8pt
    {\fontsize{7pt}{8pt}\selectfont\sffamily \labdocente}
\end{flushleft}
\thispagestyle{firststyle}


% -------------------------
% Contenido del laboratorio
% -------------------------
\begin{multicols}{2}

\section{OBJETIVOS}
\begin{enumerate}
    \item Determinar experimentalmente el valor de la aceleración de la gravedad local mediante un péndulo simple.
    \item Aplicar el método de mínimos cuadrados para ajustar datos experimentales.
    \item Analizar las fuentes de incertidumbre en la medición del período.
\end{enumerate}
\section{FUNDAMENTO TEÓRICO}
Para un conjunto de datos $(x_i, y_i)$ con $i=1,\dots,N$ y suponiendo un modelo lineal
\begin{equation*}
y_i = a + b x_i,
\end{equation*}
los parámetros $a$ y $b$ se pueden determinar de forma **independiente** mediante:

\begin{equation*}
b = \frac{N \sum_i x_i y_i - \sum_i x_i \sum_i y_i}{N \sum_i x_i^2 - (\sum_i x_i)^2}, \quad
a = \frac{\sum_i y_i - b \sum_i x_i}{N}.
\end{equation*}

Las incertidumbres asociadas se calculan como:

\begin{equation*}
\sigma_b = \sqrt{\frac{\sum_i (y_i - a - b x_i)^2}{N \sum_i x_i^2 - (\sum_i x_i)^2}}, \quad
\sigma_a = \sqrt{\frac{\sum_i x_i^2 \sum_i (y_i - a - b x_i)^2}{N \left(N \sum_i x_i^2 - (\sum_i x_i)^2\right)}}.
\end{equation*}

\medskip
\textbf{Linealización de ecuaciones no lineales:} Para relaciones físicas no lineales, como la ecuación de Torricelli
\begin{equation*}
v = \sqrt{2 g h},
\end{equation*}
se puede linealizar tomando logaritmos:

\begin{equation*}
\ln v = \frac{1}{2} \ln h + \frac{1}{2} \ln (2 g),
\end{equation*}

lo que permite aplicar un ajuste lineal de $\ln v$ vs $\ln h$ para determinar la pendiente y la ordenada de manera directa, y a partir de ellas calcular la constante de la gravedad $g$.
\section{PRÁCTICA 01: PÉNDULO SIMPLE Y AJUSTE LINEAL}

\subsection{Objetivos Específicos}
\begin{enumerate}
    \item Medir el período de oscilación de un péndulo simple para diferentes longitudes.
    \item Determinar $g$ mediante ajuste lineal y propagación de incertidumbres.
    \item Analizar la relación funcional $T(l)$ y linealizar los datos mediante logaritmos.
\end{enumerate}

\subsection{Fundamento Teórico}
El período de un péndulo simple para pequeñas amplitudes está dado por:
\begin{equation}
T = 2 \pi \sqrt{\frac{l}{g}}
\end{equation}
Linealizando mediante logaritmos:
\begin{equation}
\log T = \log(2 \pi) + \frac{1}{2} \log l - \frac{1}{2} \log g
\end{equation}
donde la pendiente de la recta permite calcular $g$ experimentalmente:
\begin{equation}
g = \left(\frac{2 \pi}{10^{b}}\right)^2
\end{equation}

\subsection{Materiales}
\begin{itemize}
    \item Péndulo simple
    \item Cronómetro (centésimas de segundo)
    \item Regla métrica (milímetros)
    \item Soporte universal
    \item Barrilla de aluminio y mordaza
\end{itemize}

\subsection{Procedimiento Experimental}
\begin{enumerate}
    \item Arme el péndulo simple y mida la longitud $L$ desde el punto de suspensión hasta el centro de masa.
    \item Realice múltiples oscilaciones (10 oscilaciones) y mida el tiempo total $t$.
    \item Calcule el período: $T = t / 10$.
    \item Repita para distintas longitudes $L$ del péndulo.
\end{enumerate}

\subsection{Recolección de Datos}
    \begin{table}[H]
        \centering
        \caption{Promedio del Periodo del péndulo para cada medida de longitud de la cuerda}
        \label{tabla:2}
        \begin{tabular}{{lcccc|>{\columncolor[gray]{0.9}}c>{\columncolor[gray]{0.9}}r}}
            \hline
            N$^\circ$ & L (m)& $\delta$ L(m) & t (s) & $\delta t$ (s) &  $T(s)$& $\delta(T)$ \\
            \hline
            1 & 1.00 & & & & &\\
            2 & 0.95 & & &&&\\
            3 & 0.90 & & &&&\\
            4 & 0.85 & & &&&\\
            5 & 0.80 & & &&&\\
            6 & 0.75 & & &&&\\
            7 & 0.70 & & &&&\\
            8 & 0.65 & & &&&\\
            9 & 0.60 & & &&&\\
            10 & 0.55 & & &&&\\
            11 & 0.50 & & &&&\\
            \hline
        \end{tabular}
    \end{table}

\subsection{Análisis de Datos}
\begin{enumerate}
    \item Realice la linealización de los datos con logaritmos y construya la tabla de sumatorias para mínimos cuadrados:

        \begin{table}[H]
        	\centering
        	\caption{Procesamiento de datos experimentales EXPERIENCIA A.}
        	\label{tabla:analisis_graficos_exp_a}
        	\begin{tabular}{lcccc|>{\columncolor[gray]{0.9}}r}
        		\hline
        		 & $x$ & $y$ & $x*y$ & $x^2$ & $(y_i-b-ax_i)^2$\\
        		$N$ & $\log(l)$ & $\log{T}$ & $\log{l}\log{T}$ & $\log{l}^2$ & $(\log{T}-b-a\log{l})^2$\\
        		\hline
        		1 &  & & & & \\
        		2 &  & & & & \\
        		3 &  & & & & \\
        		4 &  & & & & \\
        		5 &  & & & & \\
        		6 &  & & & & \\
        		7 &  & & & & \\
        		8 &  & & & & \\
        	    9 &  & & & & \\
        		10 & & & & & \\
        		11 & & & & & \\ \hline
        		$\sum$ & & & & & \\
        		\hline
        	\end{tabular}
        \end{table}

    \item Determine la pendiente $a$, la ordenada $b$, y calcule $g$ junto a su incertidumbre.
    \item Grafique $T$ vs $L$ y superponga la curva ajustada.
\end{enumerate}


\subsection{Preguntas de Análisis}
\begin{enumerate}
    \item ¿Cuál es el valor experimental de $g$ y cómo se compara con el teórico?
    \item ¿Qué fuentes de error afectan más los resultados?
    \item ¿Por qué es útil linealizar la ecuación del péndulo para aplicar mínimos cuadrados?
\end{enumerate}

\section{PRÁCTICA 02: ECUACIÓN DE TORRICELLI Y DETERMINACIÓN DE $g$ CON FOTOPUERTA}

\subsection{Objetivos Específicos}
\begin{enumerate}
    \item Medir la velocidad de salida de un chorro de agua desde distintas alturas del fluido usando una fotopuerta.
    \item Determinar la aceleración de la gravedad $g$ mediante un ajuste lineal de mínimos cuadrados.
    \item Aplicar la linealización logarítmica de la ecuación de Torricelli para simplificar el análisis.
\end{enumerate}

\subsection{Fundamento Teórico}

La ecuación de Torricelli establece:

\begin{equation*}
v = \sqrt{2 g h}
\end{equation*}

donde $v$ es la velocidad de salida del chorro y $h$ la altura del fluido sobre el orificio.  

Para linealizar esta relación y usar mínimos cuadrados, aplicamos logaritmos:

\begin{equation*}
\ln v = \frac{1}{2} \ln(2 g) + \frac{1}{2} \ln h
\end{equation*}

De esta forma, si definimos:

\begin{equation*}
Y = \ln v, \quad X = \ln h
\end{equation*}

obtenemos una ecuación de recta:

\begin{equation*}
Y = a + b X
\end{equation*}

donde:

\begin{equation*}
b = \frac{1}{2}, \quad a = \frac{1}{2} \ln(2 g)
\end{equation*}

Luego, a partir de la pendiente $a$ del ajuste lineal:

\begin{equation*}
g_\text{exp} = \frac{e^{2 a}}{2}
\end{equation*}

Para determinar la velocidad del chorro de agua que sale del recipiente, se puede usar el sensor de fotopuerta de la siguiente manera:

\begin{itemize}
    \item \textbf{Con una fotopuerta:} Coloque la fotopuerta en la salida del orificio. El sensor detecta el paso del chorro y permite medir el tiempo $t$ que tarda en interrumpir el haz de luz. A partir de este tiempo y el diámetro del chorro, se puede calcular la velocidad promedio $v$:
    \begin{equation*}
        v = \frac{d}{t}
    \end{equation*}
    donde $d$ es el diámetro efectivo del chorro o la distancia de referencia que atraviesa la fotopuerta.
    
    \item \textbf{Con dos fotopuertas:} Si se colocan dos fotopuertas separadas una distancia conocida $L$, se puede medir el tiempo $\Delta t$ que tarda el chorro en ir de la primera a la segunda:
    \begin{equation*}
        v = \frac{L}{\Delta t}
    \end{equation*}
    Este método permite una medición más precisa y elimina la necesidad de estimar el diámetro del chorro.
\end{itemize}

\subsection{Materiales}
\begin{itemize}
    \item Recipiente con agua
    \item Soporte para el recipiente
    \item Fotopuerta y sensor Pasco Capstone
    \item Computadora con Pasco Capstone
    \item Regla o cinta métrica
\end{itemize}

\subsection{Procedimiento Experimental}
\begin{enumerate}
    \item Coloque el recipiente con agua sobre el soporte.
    \item Configure la fotopuerta en el orificio de salida del líquido de forma que el chorro interrumpa el haz de luz al pasar.
    \item Configure Capstone para medir el tiempo $\Delta t$ que tarda el chorro en pasar por la fotopuerta.
    \item Abra el orificio a una altura $h_1$ y registre el tiempo $\Delta t_1$ de paso.
    \item Repita para varias alturas del fluido $h_i$, tomando al menos 5–10 mediciones por altura para reducir error.
    \item Calcule la velocidad de salida usando $v_i = d / \Delta t_i$, donde $d$ es el ancho del chorro (o el diámetro del haz de fotopuerta).
    \item Finalmente, construya la tabla de logaritmos $\ln v_i$ y $\ln h_i$ para realizar el ajuste lineal.
\end{enumerate}

\subsection{Recolección de Datos}

\begin{table}[H]
    \centering
    \caption{Recolección de datos: velocidad del chorro vs altura del fluido}
    \label{tabla:torricelli}
    \begin{tabular}{lcc|cc|c}
        \hline
        N$^\circ$ & $h$ (m) & $\delta h$ (m) & $t$ (s) & $\delta t$ (s) & $v$ (m/s)\\
        \hline
        1 & & & & & \\
        2 & & & & & \\
        3 & & & & & \\
        4 & & & & & \\
        5 & & & & & \\
        6 & & & & & \\
        7 & & & & & \\
        8 & & & & & \\
        9 & & & & & \\
        10 & & & & & \\
        \hline
    \end{tabular}
\end{table}


\subsection{Análisis de Datos}
\begin{enumerate}
    \item Grafique $\ln v$ vs $\ln h$ y realice un ajuste lineal por mínimos cuadrados.
    \item Determine la ordenada $a$ de la recta.
        \begin{table}[H]
	\centering
	\caption{Procesamiento de datos experimentales EXPERIENCIA B: Torricelli.}
	\label{tabla:analisis_graficos_exp_b}
	\begin{tabular}{lcccc|>{\columncolor[gray]{0.9}}r}
		\hline
		 & $x$ & $y$ & $x*y$ & $x^2$ & $(y_i-b-ax_i)^2$\\
		$N$ & $\log(h)$ & $\log(v)$ & $\log(h)\log(v)$ & $\log(h)^2$ & $(\log(v)-b-a\log(h))^2$\\
		\hline
		1 &  & & & & \\
		2 &  & & & & \\
		3 &  & & & & \\
		4 &  & & & & \\
		5 &  & & & & \\
		6 &  & & & & \\
		7 &  & & & & \\
		8 &  & & & & \\
		9 &  & & & & \\
		10 & & & & & \\
		\hline
		$\sum$ & & & & & \\
		\hline
	\end{tabular}
\end{table}

    \item Calcule $g_\text{exp}$ usando:

    \begin{equation*}
    g_\text{exp} = \frac{e^{2 a}}{2}
    \end{equation*}

    \item Compare el valor experimental con el valor teórico de $g$.
\end{enumerate}

\subsection{Preguntas de Análisis}
\begin{enumerate}
    \item ¿El valor experimental de $g$ coincide con el teórico?  
    \item ¿Qué ventajas ofrece la fotopuerta frente al método de cronometraje manual?  
    \item ¿Cómo afecta la precisión del diámetro del chorro en la determinación de $v$ y $g$?  
    \item ¿Por qué es útil linealizar la ecuación de Torricelli mediante logaritmos para aplicar mínimos cuadrados?  
\end{enumerate}
\section{CONCLUSIONES}
Compare los resultados obtenidos con la teoría, evalúe la confiabilidad de los datos y comente mejoras posibles para futuras prácticas.

\section{REFERENCIAS}
\begin{itemize}
    \item Serway, Física para Ciencias e Ingeniería
    \item Tipler, Física
    \item Artículos sobre método de mínimos cuadrados en física experimental
\end{itemize}

\section{APÉNDICES}
Agregue aquí el código utilizado para su análisis de datos en Python.
\end{multicols}
