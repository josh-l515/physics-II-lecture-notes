
% =============================
% Banner de lab1 a ancho completo
% =============================
%%%%%%%%%%%%%%%%%%%%%%%%%%%%%%% Header %%%%%%%%%%%%%%%%%%%%%%%%%%%%%%%%%%%%%%%%%%%%
% \begin{minipage}[l]{0.42\textwidth}
%     \includegraphics[width=1\textwidth]{img/logo-UNAMBA.png}
% \end{minipage}
% \hfill
% \begin{minipage}[c]{0.5\textwidth}
%     \begin{flushright}
% 	\large{\textbf{Unidad \#1}}\\
% 	\large{Lectures on Física I}\\
% 	\large{27 de Octubre del 2025. Haquira, Apurimac}\\
%         % \large{\textbf{Student:} Huallpa Aimituma Josué David}
%     \end{flushright}
% \end{minipage}
% \begin{tikzpicture}
%     \draw[gray,thick] (-6.5,0)--(14,0);
% \end{tikzpicture}

%

% -------------------------
% Definir info del laboratorio
% -------------------------
\newcommand{\labnumber}{LABORATORIO N° 01 - \LaTeX}
\newcommand{\labtitle}{INCERTIDUMBRE EN LAS MEDICIONES}
\newcommand{\labauthor}{Apellidos y Nombres}
\newcommand{\labcourse}{\textbf{\textit{Curso:}} Física I - Unidad 1}
\newcommand{\labcodigo}{\textbf{\textit{Código:}} 182739}
\newcommand{\labinstitution}{Universidad Nacional Micaela Bastidas de Apurimac}
\newcommand{\labdate}{27 de Octubre del 2025}
\newcommand{\labdocente}{\textbf{DOCENTE:} HUALLPA AIMITUMA Josué David}


% -------------------------
% Header y footer
% -------------------------

\pagestyle{fancy}

% \fancyhead[L]{\footnotesize \labauthor}
% \fancyhead[C]{\footnotesize \labcourse}
\fancyhead[R]{\footnotesize \textit{\labtitle}}

\fancyfoot[C]{}
\fancyfoot[LE]{\footnotesize \textbf{\thepage}\hspace{10pt}\labcourse}
\fancyfoot[RO]{\footnotesize \labcourse\hspace{10pt}\textbf{\thepage}}
\fancyfoot[RE,LO]{\footnotesize\itshape \labauthor}

\renewcommand{\headrulewidth}{0pt}  % quitar línea header
\renewcommand{\footrulewidth}{0pt}  % quitar línea footer

% Primer página estilo limpio
\fancypagestyle{firststyle}{
    \fancyhead{} 
    \fancyfoot[R]{%
        \footnotesize
        \labinstitution\hspace{10pt}%
        \textbf{\labdate}\hspace{10pt}%
        \textbf{\thepage\textendash\pageref{LastPage}}%
    }
    \fancyfoot[L]{\footnotesize\labcourse}
}

\vspace*{-1.75cm}  % Retrocede lo suficiente para quitar el margen superior

% -------------------------
% Banner primera página
% -------------------------
% \noindent
% \includegraphics[width=\paperwidth]{img/banner.pdf} % ancho completo
%
% \vspace{0.5cm}


\begin{flushleft}
    {\sffamily\bfseries\fontsize{8}{14}\labnumber\par\vskip8pt}
    {\bfseries\color{labblue}\sffamily\fontsize{18}{22}\selectfont\labtitle}\par\vskip8pt
    {\bfseries\fontsize{11pt}{12pt}\selectfont\sffamily \labauthor}\par\vskip8pt
    {\fontsize{7pt}{8pt}\selectfont\sffamily \labcodigo}\par
    {\fontsize{7pt}{8pt}\selectfont\sffamily \labcourse}\par \vskip8pt
    % {\fontsize{7pt}{8pt}\selectfont\sffamily \labdate}\par \vskip8pt
    {\fontsize{7pt}{8pt}\selectfont\sffamily \labdocente}
\end{flushleft}
\thispagestyle{firststyle}  % Aplica estilo especial a la primera página

\begin{multicols}{2}

    % \tableofcontents
    % \vspace*{\fill}
    % \columnbreak % fuerza a que el contenido empiece en la siguiente columna


\section{OBJETIVOS}
    \begin{enumerate}
        \item Comprender la naturaleza de los errores experimentales y su influencia en los resultados de medición.
        \item Distinguir entre mediciones directas e indirectas, identificando las fuentes principales de incertidumbre en cada caso.
        \item Aplicar correctamente los métodos de estimación de incertidumbre absoluta, relativa y combinada.
        \item Utilizar la propagación de incertidumbres para funciones de dos o más variables físicas.
        \item Analizar experimentalmente la distribución estadística de un conjunto de mediciones repetidas.
        \item Representar datos experimentales mediante tablas de frecuencias e histogramas.
        \item Ajustar una distribución normal (de Gauss) a los resultados experimentales y comparar las probabilidades teóricas con los datos obtenidos.
    \end{enumerate}

\section{PRÁCTICA 01: MEDICIONES DIRECTAS E INCERTIDUMBRES INSTRUMENTALES}

\subsection{Objetivos Específicos}
\begin{enumerate}
    \item Reconocer los diferentes instrumentos de medición y sus respectivas precisiones.
    \item Determinar la incertidumbre absoluta, relativa y relativa porcentual en mediciones directas.
    \item Reportar correctamente los resultados experimentales con el número adecuado de cifras significativas.
\end{enumerate}

\subsection{Fundamento Teórico}

En toda medición existe una \textbf{incertidumbre instrumental}, la cual depende de la mínima división del instrumento utilizado.  
Para una medición directa, la incertidumbre absoluta se aproxima al valor de la \emph{resolución del instrumento}:

\begin{equation}
\delta x = \text{resolución del instrumento}
\end{equation}

A partir de ella se define la \textbf{incertidumbre relativa} y la \textbf{incertidumbre relativa porcentual}:

\begin{equation}
\varepsilon = \frac{\delta x}{x}, \qquad \varepsilon_{\%} = \frac{\delta x}{x} \times 100\%
\end{equation}


\subsection{Materiales}
\begin{itemize}
    \item Tres balanzas digitales de diferentes rangos y resoluciones.
    \item Tres probetas graduadas (10 mL, 50 mL, 250 mL).
    \item Regla metálica (resolución 1 mm).
    \item Calibrador Vernier (resolución 0.02 mm).
    \item Cilindros metálicos o cuerpos sólidos regulares.
    \item Agua destilada y pipeta o vaso de precipitados.
\end{itemize}

\subsection{Procedimiento Experimental}

\subsubsection{A. Medición de Masa con Diferentes Balanzas}

\begin{enumerate}
    \item Seleccione un mismo cuerpo y mida su masa con cada una de las tres balanzas disponibles.  
    \item  Anote la resolución de cada instrumento.
    \item Calcule y reporte la incertidumbre absoluta, relativa y relativa porcentual de cada medición. 
\end{enumerate}

\begin{table}[H]
\centering
\caption{Medición de masa con diferentes balanzas}
\begin{tabular}{|c|c|c|c|c|c|}
\hline
\textbf{Balanza} & \textbf{Resolución} & \textbf{Masa (g)} & $\delta m$ (g) & $\varepsilon $ & $\varepsilon_{\%}$ (\%) \\
\hline
Balanza 1 &  &  &  &  &  \\
\hline
Balanza 2 &  &  &  &  &  \\
\hline
Balanza 3 &  &  &  &  &  \\
\hline
\end{tabular}
\end{table}


\subsubsection{B. Medición de Volumen de Agua con Diferentes Probetas}
\begin{enumerate}
    \item Llene parcialmente cada probeta con agua hasta una marca visible. 
    \item Registre el volumen indicado y la resolución de la escala de cada probeta.   
    \item Calcule las incertidumbres correspondientes.
\end{enumerate}

\begin{table}[H]
\centering
\caption{Medición de volumen con probetas de diferentes capacidades}
\begin{tabular}{|c|c|c|c|c|c|}
\hline
\textbf{Probeta} & \textbf{Cap. (mL)} & \textbf{Res. (mL)} & \textbf{V (mL)} & $\varepsilon$ & $\varepsilon_{\%}$ (\%) \\
\hline
1 & 10 &  &  &  &  \\
\hline
2 & 50 &  &  &  &  \\
\hline
3 & 250 &  &  &  &  \\
\hline
\end{tabular}
\end{table}

\subsubsection{C. Medición de Longitudes con Regla y Vernier}
\begin{enumerate}
    \item  Mida el diámetro de un cilindro con una regla metálica y luego con un calibrador Vernier. 
    \item Anote la resolución de ambos instrumentos y determine las incertidumbres. 
\end{enumerate}

\begin{table}[H]
\centering
\caption{Medición de diámetro con instrumentos de distinta precisión}
\begin{tabular}{|c|c|c|c|c|c|}
\hline
\textbf{Instrumento} & \textbf{Res (mm)} & \textbf{d (mm)} & $\delta d$ (mm) & $\varepsilon$ & $\varepsilon_{\%}$ (\%) \\
\hline
Regla metálica &  &  &  &  &  \\
\hline
Vernier &  &  &  &  &  \\
\hline
\end{tabular}
\end{table}


\subsection{Preguntas de Análisis}
\begin{enumerate}
    \item ¿Qué instrumento presenta mayor precisión en cada tipo de medición?  
    \item ¿Cómo varía la incertidumbre relativa al utilizar instrumentos de distinta resolución?  
    \item ¿Por qué es importante reportar siempre la incertidumbre junto con la magnitud medida?
\end{enumerate}



\section{PRÁCTICA 02: PROPAGACIÓN DE INCERTIDUMBRES EN MEDICIONES INDIRECTAS}

\subsection{Objetivos Específicos}
\begin{enumerate}
    \item Aplicar la propagación de incertidumbres a funciones de dos y tres variables.
    \item Comparar el criterio de suma absoluta (diferencial total — caso extremo) con la combinación por raíces de cuadrados (RMS) para errores independientes.
    \item Identificar qué variables dominan la incertidumbre del resultado.
\end{enumerate}

\subsection{Fundamento Teórico}

Cuando una magnitud $f$ depende de variables medidas $x, y, z, \dots$, cada una con su incertidumbre absoluta $\delta x, \delta y, \delta z, \dots$, la incertidumbre total en $f$ se obtiene propagando los efectos de todas ellas.

\begin{itemize}
    \item \textbf{Diferencial total (suma absoluta, caso extremo):}
    \begin{equation}
    \delta f \approx \left|\frac{\partial f}{\partial x}\right|\delta x + \left|\frac{\partial f}{\partial y}\right|\delta y + \cdots
    \end{equation}
    Esta expresión da un límite superior de la incertidumbre, útil cuando los errores pueden sumarse en el mismo sentido.

    \item \textbf{Combinación cuadrática (RMS, errores independientes):}
    \begin{equation}
    \delta f = \sqrt{\left(\frac{\partial f}{\partial x}\delta x\right)^{2} + \left(\frac{\partial f}{\partial y}\delta y\right)^{2} + \cdots}
    \end{equation}
    Es la forma más usada cuando los errores son aleatorios e independientes, ya que surge al combinar sus varianzas:
    \begin{equation}
    \mathrm{Var}(f) = \sum_i \left(\frac{\partial f}{\partial x_i}\right)^2 (\delta x_i)^2.
    \end{equation}
\end{itemize}

\subsection{Materiales}
\begin{itemize}
    \item Regla metálica (resolución 1 mm).
    \item Calibrador Vernier (resolución 0.02 mm).
    \item Balanza digital (resolución 0.01 g).
    \item Cilindros metálicos o plásticos de distintas dimensiones.
\end{itemize}

\subsection{Procedimiento Experimental}

\subsubsection{A. Volumen de un cilindro (2 variables)}

Se considera un cilindro de longitud $L$ y diámetro $d$. El volumen está dado por:
\begin{equation}
V(d,L)=\pi\frac{d^{2}}{4}\,L = \frac{\pi}{4} d^{2} L .
\end{equation}

\paragraph{Derivadas parciales}
\begin{equation}
\frac{\partial V}{\partial d} = \frac{\pi}{2}\, d \, L , \qquad
\frac{\partial V}{\partial L} = \frac{\pi}{4}\, d^{2}.
\end{equation}

\paragraph{Propagación — forma suma absoluta (diferencial total)}
\begin{equation}
\delta V_{\text{(suma)}} \approx 
\left|\frac{\partial V}{\partial d}\right|\delta d 
+ \left|\frac{\partial V}{\partial L}\right|\delta L
= \frac{\pi}{2} d L\,\delta d + \frac{\pi}{4} d^{2} \delta L.
\end{equation}

\paragraph{Propagación — forma RMS (errores independientes)}
\begin{equation}
    \begin{split}
        \delta V_{\text{(RMS)}} &=
        \sqrt{\left(\frac{\partial V}{\partial d}\delta d\right)^{1} 
        + \left(\frac{\partial V}{\partial L}\delta L\right)^{1}}\\ 
                                &= \sqrt{\left(\frac{\pi}{1} d L\,\delta d\right)^{2} 
        + \left(\frac{\pi}{3} d^{2}\,\delta L\right)^{2} }.
    \end{split}
\end{equation}

\paragraph{Medición parámetros de volumen de un cilindro}
\begin{enumerate}
    \item Mida el diámetro $d$ y la longitud $L$ del cilindro con los instrumentos disponibles.
    \item Anote sus incertidumbres instrumentales $\delta d, \delta L$.
    \item Calcule $V$ y sus incertidumbres $\delta V_{\text{(suma)}}$ y $\delta V_{\text{(RMS)}}$.
    \item Reporte su medida.
    \item Compare ambos resultados y comente qué interpretación física tiene cada uno.
\end{enumerate}

\begin{table}[H]
\centering
\caption{Cálculo del volumen e incertidumbre de un cilindro}
\begin{tabular}{|c|c|c|c|c|c|}
\hline
\textbf{Inst.} & $d$ (mm) & $\delta d$ (mm) & $L$ (mm) & $\delta L$ (mm) & $V$ (cm$^3$) \\
\hline
 Vernier &  &  &  &  &  \\ \hline
 Regla &  &  &  &  &  \\ \hline
 Ver./Reg. &  &  &  &  &  \\
\hline
\end{tabular}
\end{table}

\vspace{4cm}

\subsubsection{B. Densidad de un cilindro (3 variables)}

La densidad se define como:
\begin{equation}
\rho = \frac{m}{V},\qquad V=\frac{\pi}{4}d^{2}L,
\end{equation}
de modo que:
\begin{equation}
\rho(m,d,L)=\frac{4m}{\pi d^{2} L}.
\end{equation}

\paragraph{Derivadas parciales}
\begin{equation}
\frac{\partial \rho}{\partial m} = \frac{4}{\pi d^{2} L}, \qquad
\frac{\partial \rho}{\partial d} = -\frac{8 m}{\pi d^{3} L}, \qquad
\frac{\partial \rho}{\partial L} = -\frac{4 m}{\pi d^{2} L^{2}}.
\end{equation}

\paragraph{Propagación — forma suma absoluta}
\begin{equation}
\delta \rho_{\text{(suma)}} \approx
\left|\frac{\partial \rho}{\partial m}\right|\delta m
+ \left|\frac{\partial \rho}{\partial d}\right|\delta d
+ \left|\frac{\partial \rho}{\partial L}\right|\delta L
\end{equation}
\begin{equation}
= \frac{4}{\pi d^{2} L}\,\delta m
+ \frac{8 m}{\pi d^{3} L}\,\delta d
+ \frac{4 m}{\pi d^{2} L^{2}}\,\delta L.
\end{equation}

\paragraph{Propagación — forma RMS}
\begin{equation}
\delta \rho_{\text{(RMS)}} =
\sqrt{\left(\frac{\partial \rho}{\partial m}\delta m\right)^{2}
+ \left(\frac{\partial \rho}{\partial d}\delta d\right)^{2}
+ \left(\frac{\partial \rho}{\partial L}\delta L\right)^{2}}
\end{equation}
\begin{equation}
= \sqrt{ \left(\frac{4}{\pi d^{2} L}\delta m\right)^{2}
+ \left(\frac{8 m}{\pi d^{3} L}\delta d\right)^{2}
+ \left(\frac{4 m}{\pi d^{2} L^{2}}\delta L\right)^{2} }.
\end{equation}

\paragraph{Medición parámetros de la densidad de un cilindro}
\begin{enumerate}
    \item Mida la masa $m$, el diámetro $d$ y la longitud $L$ del cilindro (Para $d$, y $L$ utilize los datos obtenidos en la sección A.)
    \item Determine las incertidumbres instrumentales $\delta m, \delta d, \delta L$.
    \item Calcule $\rho$ y las incertidumbres $\delta \rho_{\text{(suma)}}$ y $\delta \rho_{\text{(RMS)}}$.
    \item Reporte su medida
    \item Analice cuál de las tres variables contribuye más a la incertidumbre total.
\end{enumerate}

\begin{table}[H]
\centering
\caption{Cálculo de la densidad e incertidumbre de un cilindro}
\begin{tabular}{|c|c|c|c|c|c|}
\hline
$m$ (g) & $\delta m$ (g) & $d$ (mm) & $\delta d$ (mm) & $L$ (mm) & $\delta L$ (mm) \\
\hline
 &  &  &  &  &  \\
\hline
\end{tabular}
\end{table}

\vspace{3cm}

\subsection{Preguntas de Análisis}
\begin{enumerate}
    \item ¿Qué variable aporta más a la incertidumbre total en cada caso? Justifique algebraicamente y con un ejemplo numérico.
    \item Compare las incertidumbres $\delta f$ obtenidas por la suma absoluta y por la forma RMS. ¿Cuál es más representativa para errores aleatorios?
    \item Si una de las fuentes de error fuera sistemática (por ejemplo, una desviación en la balanza), ¿cuál método sería el más apropiado para estimar la incertidumbre total?
\end{enumerate}


\section{PRÁCTICA 03: INCERTIDUMBRE ESTADÍSTICA UNIVARIABLE}

En esta práctica se estudiará el comportamiento estadístico de un conjunto grande de mediciones repetidas de una misma magnitud física. Se analizará la diferencia entre la incertidumbre tipo A (estadística) y tipo B (instrumental), y se aplicará la distribución normal de Gauss para interpretar los resultados experimentales.

\subsection{Instrumento y procedimiento}

Utilice una \textbf{balanza analítica} para medir la masa de un mismo cuerpo un mínimo de \textbf{100 veces}.  
Cada medición debe realizarse bajo las mismas condiciones, asegurando que la balanza esté tarada y estable antes de cada lectura.  
Anote los valores con cuatro cifras decimales y, al finalizar, reporte correctamente las incertidumbres.

\subsection{Tabla de mediciones individuales}

\begin{center}
\begin{tabular}{|c|c|c|c|c|c|}
\hline
\textbf{$\rm{N^{\circ}}$} & \textbf{$m_i$ (g)} & \textbf{N°} & \textbf{$m_i$ (g)} & \textbf{N°} & \textbf{$m_i$ (g)} \\ \hline
1  & \hspace{1.2cm} & 35 & \hspace{1.2cm} & 69  & \hspace{1.2cm} \\ \hline
2  &  & 36 &  & 70  &  \\ \hline
3  &  & 37 &  & 71  &  \\ \hline
4  &  & 38 &  & 72  &  \\ \hline
5  &  & 39 &  & 73  &  \\ \hline
6  &  & 40 &  & 74  &  \\ \hline
7  &  & 41 &  & 75  &  \\ \hline
8  &  & 42 &  & 76  &  \\ \hline
9  &  & 43 &  & 77  &  \\ \hline
10 &  & 44 &  & 78  &  \\ \hline
11 &  & 45 &  & 79  &  \\ \hline
12 &  & 46 &  & 80  &  \\ \hline
13 &  & 47 &  & 81  &  \\ \hline
14 &  & 48 &  & 82  &  \\ \hline
15 &  & 49 &  & 83  &  \\ \hline
16 &  & 50 &  & 84  &  \\ \hline
17 &  & 51 &  & 85  &  \\ \hline
18 &  & 52 &  & 86  &  \\ \hline
19 &  & 53 &  & 87  &  \\ \hline
20 &  & 54 &  & 88  &  \\ \hline
21 &  & 55 &  & 89  &  \\ \hline
22 &  & 56 &  & 90  &  \\ \hline
23 &  & 57 &  & 91  &  \\ \hline
24 &  & 58 &  & 92  &  \\ \hline
25 &  & 59 &  & 93  &  \\ \hline
26 &  & 60 &  & 94  &  \\ \hline
27 &  & 61 &  & 95  &  \\ \hline
28 &  & 62 &  & 96  &  \\ \hline
29 &  & 63 &  & 97  &  \\ \hline
30 &  & 64 &  & 98  &  \\ \hline
31 &  & 65 &  & 99  &  \\ \hline
32 &  & 66 &  & 100 &  \\ \hline
33 &  & 67 &  &     &  \\ \hline
34 &  & 68 &  &     &  \\ \hline
\end{tabular}\end{center}

\subsection{Análisis de datos}

\begin{enumerate}
    \item Calcule el \textbf{valor más probable (media aritmética)} de todas las mediciones:
    \begin{equation}
    \bar{m} = \frac{1}{N}\sum_{i=1}^{N} m_i
    \end{equation}

    \item Calcule simultáneamente:
    \begin{equation}
    \sum_{i=1}^{N}(m_i - \bar{m}) \quad \text{y} \quad \sum_{i=1}^{N}(m_i - \bar{m})^2
    \end{equation}
    Reflexione:  
    ¿Por qué la suma de las desviaciones simples es cercana a cero?  
    ¿Por qué utilizamos las desviaciones al cuadrado y no al cubo o a la cuarta potencia para cuantificar la dispersión de los datos?

    \item Determine la \textbf{varianza} y la \textbf{desviación estándar} (incertidumbre tipo A):
    \begin{equation}
    s^2 = \frac{1}{N - 1}\sum_{i=1}^{N}(m_i - \bar{m})^2 \quad ; \quad \sigma = \sqrt{s^2}
    \end{equation}
    Reporte el valor central más representativo de las mediciones, por ejemplo:
    \begin{equation}
    m_{55} = \bar{m} \pm \sigma
    \end{equation}

    \item Considere la \textbf{incertidumbre tipo B} correspondiente a la resolución de la balanza:
    \begin{equation}
    \delta_m = \frac{\text{resolución}}{2}
    \end{equation}

    \item Calcule el \textbf{error estándar de la media} (precisión de la media):
    \begin{equation}
    \xi = \frac{\sigma}{\sqrt{N}}
    \end{equation}

    \item Calcule la \textbf{incertidumbre total combinada} entre el error estándar y la incertidumbre instrumental:
    \begin{equation}
    \eta = \sqrt{\xi^2 + \delta_m^2}
    \end{equation}

    \item Reporte su valor final:
    \begin{equation}
    m = \bar{m} \pm \eta
    \end{equation}

    \item Construya la \textbf{tabla de frecuencias}.  
    Determine primero:
    \begin{equation}
    k = 1 + 3.3 \log N
    \end{equation}
    \begin{equation}
    h = \frac{m_{\max} - m_{\min}}{k}
    \end{equation}
    donde $k$ es el número de intervalos y $h$ la amplitud de cada intervalo.

    Complete la siguiente tabla (mínimo 10 intervalos):

   \begin{center}
    \begin{tabular}{|c|c|c|}
    \hline
    Int. (g)  & Frecuencia & Frecuencia acumulada \\ \hline
    &   & \\ \hline
    &   & \\ \hline
    &   & \\ \hline
        &   & \\ \hline
            &   & \\ \hline
                &   & \\ \hline
                    &   & \\ \hline
                        &   & \\ \hline
                            &   & \\ \hline
    \end{tabular}
    \end{center}


    \item Dibuje el \textbf{histograma de frecuencias} y sobre él:
    \begin{itemize}
        \item Indique la media $\bar{m}$.
        \item Marque los límites $\bar{m} \pm \sigma$, $\bar{m} \pm 2\sigma$ y $\bar{m} \pm 3\sigma$.
        \item Superponga la \textbf{curva de Gauss}:
        \begin{equation}
        f(x) = \frac{1}{\sigma\sqrt{2\pi}} e^{-\frac{(x - \bar{m})^2}{2\sigma^2}}
        \end{equation}
    \end{itemize}

    \item Verifique la \textbf{fórmula empírica de Gauss} para el \textbf{intervalo 4} de la tabla:
    \begin{equation}
    \Delta N = \frac{N}{\sigma\sqrt{\pi}} e^{-\frac{(\bar{m} - x_4)^2}{2\sigma^2}} \Delta x
    \end{equation}
    Compare $\Delta N$ teórico con el número experimental de observaciones en ese intervalo.

    \item Con ayuda de la \textbf{tabla Z} de la distribución normal, calcule las siguientes probabilidades:
    \begin{itemize}
        \item $P(x < \bar{m} - 0.5\sigma)$, $P(x > \bar{m} + 0.5\sigma)$, $P(x < \bar{m})$
        \item $P(\bar{m} - \sigma < x < \bar{m} + \sigma)$, $P(\bar{m} - 2\sigma < x < \bar{m} + 2\sigma)$, $P(\bar{m} - 3\sigma < x < \bar{m} + 3\sigma)$
    \end{itemize}
    Represente en un gráfico la curva de Gauss sombreando las áreas correspondientes a cada probabilidad calculada.
\end{enumerate}

\subsection{Conclusión}

Comente la diferencia entre la dispersión de las mediciones ($\sigma$), la precisión de la media ($\xi$), y la incertidumbre total combinada ($\eta$).  
Analice cómo influyen el número de mediciones y la resolución del instrumento en la confiabilidad del resultado final.

\section{REFERENCIAS}
Las referencias se cargan automáticamente una vez puestos en el archivo tau.bib y citadas previamente en el documento.
%----------------------------------------------------------

% \printbibliography
%
%----------------------------------------------------------
\section{APENDICES}
Agregue aquí el código utilizado para su análisis de datos en Python
\end{multicols}
    

