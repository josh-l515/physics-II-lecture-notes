\section{Leyes de Maxwell-wilson}

Las \textbf{leyes de Maxwell} constituyen el conjunto fundamental de ecuaciones que describen el comportamiento de los campos eléctricos y magnéticos, así como su relación con las cargas y corrientes eléctricas. Estas ecuaciones unifican los fenómenos eléctricos y magnéticos dentro de un mismo marco teórico, dando origen al electromagnetismo clásico.

\subsection*{Forma diferencial de las ecuaciones de Maxwell}

\begin{enumerate}
    \item \textbf{Ley de Gauss para el campo eléctrico:}
    \[
    \nabla \cdot \mathbf{E} = \frac{\rho}{\varepsilon_0}
    \]
    Indica que las cargas eléctricas son la fuente del campo eléctrico.

    \item \textbf{Ley de Gauss para el campo magnético:}
    \[
    \nabla \cdot \mathbf{B} = 0
    \]
    Expresa que no existen monopolos magnéticos, es decir, las líneas del campo magnético siempre son cerradas.

    \item \textbf{Ley de Faraday de la inducción:}
    \[
    \nabla \times \mathbf{E} = -\frac{\partial \mathbf{B}}{\partial t}
    \]
    Muestra cómo un campo magnético variable genera un campo eléctrico inducido.

    \item \textbf{Ley de Ampère-Maxwell:}
    \[
    \nabla \times \mathbf{B} = \mu_0 \mathbf{J} + \mu_0 \varepsilon_0 \frac{\partial \mathbf{E}}{\partial t}
    \]
    Explica cómo los campos magnéticos son producidos tanto por corrientes eléctricas como por campos eléctricos variables en el tiempo.
\end{enumerate}

\subsection*{Importancia}

Estas
