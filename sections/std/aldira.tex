\section{Oscilaciones}

\subsection{Introducción}

Las oscilaciones son movimientos periódicos de un sistema físico alrededor de una posición de equilibrio. Estos fenómenos aparecen en numerosos campos de la física, como en el movimiento de resortes, circuitos eléctricos, péndulos o incluso en las vibraciones de átomos dentro de un sólido. 

Un sistema que presenta oscilaciones conserva energía alternando entre energía cinética y potencial.

\subsection{Oscilador armónico simple}

El ejemplo más clásico de oscilación es el \textbf{oscilador armónico simple}, que describe el movimiento de una partícula sometida a una fuerza restauradora proporcional a su desplazamiento:

\[
F = -k x
\]

donde:
\begin{itemize}
    \item \( F \) es la fuerza restauradora,
    \item \( k \) es la constante elástica,
    \item \( x \) es el desplazamiento respecto al equilibrio.
\end{itemize}

Aplicando la segunda ley de Newton:

\[
m \frac{d^2x}{dt^2} = -k x
\]

lo que se reescribe como una ecuación diferencial:

\[
\frac{d^2x}{dt^2} + \omega^2 x = 0
\]

con \( \omega = \sqrt{\frac{k}{m}} \), denominada \textbf{frecuencia angular}.

\subsection{Solución de la ecuación del movimiento}

La solución general para la posición en función del tiempo es:

\[
x(t) = A \cos(\omega t + \phi)
\]

donde:
\begin{itemize}
    \item \( A \): amplitud del movimiento,
    \item \( \omega \): frecuencia angular (\( \mathrm{rad/s} \)),
    \item \( \phi \): fase inicial,
    \item \( t \): tiempo.
\end{itemize}

La \textbf{velocidad} y la \textbf{aceleración} se obtienen derivando respecto al tiempo:

\[
v(t) = \frac{dx}{dt} = -A \omega \sin(\omega t + \phi)
\]

\[
a(t) = \frac{d^2x}{dt^2} = -A \omega^2 \cos(\omega t + \phi)
\]

\subsection{Energía en el oscilador armónico simple}

La energía total \( E \) se conserva y es la suma de la energía cinética y la energía potencial:

\[
E = \frac{1}{2} m v^2 + \frac{1}{2} k x^2
\]

Sustituyendo las expresiones para \( x(t) \) y \( v(t) \), se demuestra que:

\[
E = \frac{1}{2} k A^2
\]

La energía permanece constante durante el movimiento, indicando que no hay pérdidas de energía.

\subsection{Periodo y frecuencia}

El \textbf{periodo} \( T \) es el tiempo que tarda la partícula en completar una oscilación completa:

\[
T = \frac{2\pi}{\omega} = 2\pi \sqrt{\frac{m}{k}}
\]

y la \textbf{frecuencia} \( f \) se define como:

\[
f = \frac{1}{T} = \frac{\omega}{2\pi}
\]

\subsection{Conclusiones}

El estudio del oscilador armónico simple constituye la base para comprender fenómenos más complejos, como las oscilaciones amortiguadas y forzadas, así como las ondas mecánicas y electromagnéticas. Las propiedades periódicas y conservativas de este sistema son fundamentales en la física clásica y moderna.


